\chapter{分布式机器视觉库}
\label{cha:distributed_vision_library}
Describe existing re-usable implementations

\section{媒体存储结构}
在大数据存储比如HDFS和对象存储高效批量处理小文件是一个问题。
这里定义小文见是文件大小小于大数据存储的block size。Hadoop
HDFS默认定义的block size是64MB。Swift和S3对象存储默认
block size是32MB。图片大小一般是10KB - 200KB。
如果直接保存图片,每个图片是一个独立的文件会
造成处理的时候得每个图片单独读取,发多个请求。
另外处理的时候Spark和Hadoop Map/Reduce也
无法同时处理多个图片在同一个任务。

目前在CloudVision通过两个方式解决高性能
批量处理小文件的问题:使用SequenceFiles和
Apache Parquet的存储格式和方式。SequenceFile提供
图片存储,Parquet提供中间结果存储结构。


\subsection{基于SequenceFile存储大量图片}
Hadoop的SequenceFile格式可以保存Key/Value对。
CloudVision使用SequenceFile保存大量的图片,Key是一个
string用来保存图片文件名字,Value是一个Array[Bytes]用来保存
图片的内容。

SequenceFile可以被分段(splittable),这样Spark或者Map/Reduce的任务
可以同时分别处理不同的分段,提高处理性能和并行化。

\begin{figure}[h]
  \centering
    \includegraphics[width=0.98\textwidth,trim=1 1 4 1,clip]{cloudvision-seqfile}
  \caption{CloudVision 保存图片的SequenceFile格式和用法}
  \label{fig:cloudvision-seqfile}
\end{figure}

在图\ref{fig:cloudvision-seqfile}有一个例子保存整个ImageNet2015数据集。
黄色表示Key保存文件名字:1.jpg和蓝色表示图片的裸的bytes。SequenceFile可以有
$N$个条目。

为了容易让用户转换和上传数据集,CloudVision自开发了一个工具。
在Listing \ref{lst:uploadseqfile}提供了几行工具主要代码。
工具的输入是一个图片目录和输出目录,它从图片目录读取所有
文件文件名和内容,创建出来一个SequenceFile在输出目录,
将每个图片文件保存到SequenceFile里。比如用户可以这样
调用工具:
upload\_to\_seqfile /home/sam/imagenet2014 swift://spark.swift/imagenet2014.hseq,
会将用户本地/home/sam/imagenet2014图片目录上传到Swift的对象存储保存一个SequenceFile,
Swift里面文件名是imagenet2014.hseq。

\begin{lstlisting}[language=Java,
                   basicstyle=\tiny,
                   showstringspaces=false,
                   caption={UploadToSequenceFile工具},
                   label={lst:uploadseqfile}]
Path outputPath = new Path(outputPath);
writer = SequenceFile.createWriter(conf, SequenceFile.Writer.file(seqFilePath), 
            SequenceFile.Writer.keyClass(Text.class),
            SequenceFile.Writer.valueClass(BytesWritable.class));
for (File file : imageFiles) {
    byte buffer[] = Files.toByteArray(file);
    writer.append(new Text(file.getName()), new BytesWritable(buffer));
}
writer.close();
\end{lstlisting}

在后续的Spark任务可以通过SparkContext.sequenceFile直接读取SequenceFile。
Spark会自动把Key,Value Pairs做成一个RDD。在章\ref{sec:feature-extraction}
会有详细的例子使用图片的SequenceFile抽取特征。

\subsection{基于Parquet通用数据结构}
Parquet是提供一个通用的数据列式存储。列式存储
按列保存数据,特别合适在批量处理使用。CloudVision
主要用Parquet来保存所有处理的结果,比如从图片抽取的
特征或者图像表示。
在CloudVision选上了Parquet的原因有:
\begin{itemize}
  \item 通用性 \\
        不同的处理平台都
        支持Parquet的格式包含Spark, Hadoop Map/Reduce和Cascading。
  \item 高效I/O \\
        因为度数据的时候不需要扫描。Parquet列式存储底层使用blocks保存,
        可以按整个block做一次IO请求读取block里面所有数据。另外可以
        只选需要读取的列,降低需要读取的大小。
  \item 高效压缩 \\
        Parquet的压缩编码能减少存储使用的大小。
  \item Spark SQL的DataFrame兼容性 \\
        Spark natively支持Parquet的格式。通过Spark SQL可以直接
        读取Parquet文件转成DataFrame,同样DataFrame可以直接
        写成Parquet文件。
\end{itemize}



\section{特征抽取}
\label{sec:feature-extraction}
Hand-crafted features SIFT, SURF, HOG,
First one to publish and succeed extracting SIFT/SURf features on Spark efficiently
Learned Features CNN


\section{特征编码与表示}
Classic BoW with sparse vecotrs
VLAD with distances and pooling same codebook as BoW
Feature Detection -> Feature Descriptors -> Codebook Generation
Feature Descriptors -> Codebook assign -> BoW -> Pooling -> 


\section{机器学习与分类器}
SVM, 
