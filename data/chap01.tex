\chapter{引言}
\label{cha:intro}


\section{研究背景}
\label{sec:background}
近年来,随着机器视觉,大数据时和云计算代的发展,开始展开新的领域和应用场景包含,
自主车,监控,医疗,交通,军事,IoT,机器人等各个不同的领域。
通过章节\ref{subsec:cv_background}介绍机器视觉的现状,
我们分析出来现在的挑战和主流的方向。
然后通过章节\ref{subsec:bigdata_background}介绍大数据与分布式处理和章节\ref{subsec:cloud_background}云计算研究现状我们介绍主流的基础施舍,
大数据和云的概念。最后通过章节\ref{subsec:bigdata_cv_background}基于大数据实现大规模的机器视觉研究现状介绍跟本论文类似研究。
机器视觉的发展对底层的数据和基础设施平台带来新的挑战,通过章节\ref{sec:challenges}问题与挑战会介绍大规模机器视觉的挑战。

\subsection{机器视觉研究现状}
\label{subsec:cv_background}

互联网公司像Facebook通过文字的数据分析理解用户的兴趣,
和爱好然后用这些信息推更有效的广告,设计新的产品
等等。数据已经开始作为互联网时代的金子。因此通过分析大
量的可视化的数据就可以带来更大的价值。同时在IoT和自主车
也需要机器通过头像检测到周边的环境。机器视觉是一个专
注与分析,处理和理解可视化数的数据的领域。可视化的数据
是图片,视频,多生物的数据等等。在最近几年机器视觉的应
用变了更广阔,主流的应用是自主车,车牌监控,机器人,物
体检测。

\subsection{大数据与分布数处理研究现状}
\label{subsec:bigdata_background}

\subsection{云计算领域研究现状}
\label{subsec:cloud_background}
NIST definition with characteristics

\subsection{基于大数据实现大规模机器视觉的研究现况}
\label{subsec:bigdata_cv_background}

\section{问题与挑战}
\label{sec:challenges}

\section{本文主要工作在与结构安排}
\label{sec:main_work}
