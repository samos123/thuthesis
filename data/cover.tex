\thusetup{
  %******************************
  % 注意:
  %   1. 配置里面不要出现空行
  %   2. 不需要的配置信息可以删除
  %******************************
  %
  %=====
  % 秘级
  %=====
  secretlevel={秘密},
  secretyear={10},
  %
  %=========
  % 中文信息
  %=========
  ctitle={CloudVision - 分布式的大规模机器视觉云平台},
  cdegree={工学硕士},
  cdepartment={软件工程系},
  cmajor={软件工程},
  cauthor={Sam Stoelinga(林海洋)},
  csupervisor={丁贵光副教授},
  %cassosupervisor={陈文光教授}, % 副指导老师
  %ccosupervisor={某某某教授}, % 联合指导老师
  % 日期自动使用当前时间,若需指定按如下方式修改:
  % cdate={超新星纪元},
  %
  % 博士后专有部分
  %cfirstdiscipline={计算机科学与技术},
  %cseconddiscipline={系统结构},
  %postdoctordate={2009年7月——2011年7月},
  %id={编号}, % 可以留空: id={},
  %udc={UDC}, % 可以留空
  %catalognumber={分类号}, % 可以留空
  %
  %=========
  % 英文信息
  %=========
  etitle={CloudVision - Distributed large-scale computer vision},
  % 这块比较复杂,需要分情况讨论:
  % 1. 学术型硕士
  %    edegree:必须为Master of Arts或Master of Science(注意大小写)
  %             “哲学、文学、历史学、法学、教育学、艺术学门类,公共管理学科
  %              填写Master of Arts,其它填写Master of Science”
  %    emajor:“获得一级学科授权的学科填写一级学科名称,其它填写二级学科名称”
  % 2. 专业型硕士
  %    edegree:“填写专业学位英文名称全称”
  %    emajor:“工程硕士填写工程领域,其它专业学位不填写此项”
  % 3. 学术型博士
  %    edegree:Doctor of Philosophy(注意大小写)
  %    emajor:“获得一级学科授权的学科填写一级学科名称,其它填写二级学科名称”
  % 4. 专业型博士
  %    edegree:“填写专业学位英文名称全称”
  %    emajor:不填写此项
  edegree={Master of Engineering},
  emajor={Software Engineering},
  eauthor={Sam Stoelinga},
  esupervisor={Associate Professor Ding Guiguang},
%  eassosupervisor={Chen Wenguang},
  % 日期自动生成,若需指定按如下方式修改:
  % edate={December, 2005}
  %
  % 关键词用“英文逗号”分割
  ckeywords={机器视觉,云计算,大数据, spark},
  ekeywords={computer vision, cloud, big data, spark}
}

% 定义中英文摘要和关键字
\begin{cabstract}
    to be done
\end{cabstract}

% 如果习惯关键字跟在摘要文字后面,可以用直接命令来设置,如下:
% \ckeywords{\TeX, \LaTeX, CJK, 模板, 论文}

\begin{eabstract}
    Computer Vision has been gaining more popularity and has found it's way into daily
    life applications such as Autonomous Vehicles. This has resulted in both larger
    datasets and more compute intensive workloads. Researchers in computer vision now
    not only need to solve algorithmic problems, but also need to dig into distributed
    computing, big data and computing infrastructure in order to solve state-of-the art
    computer vision problems. Basic steps such as feature extraction, dictionary learning,
    feature coding/pooling and implementing common classifiers shouldn't have to be
    implemented again and again by researchers. We present CloudVision, a cloud platform
    for Computer Vision users and researchers to efficiently run large-scale computer
    vision applications in a distributed fashion on demand.

    At the core, an horizontally scale-able architecture is used. This is necessary to
    meet the demands of large and increasing datasets. Hadoop Distributed File System(HDFS)
    or Hadoop Compatible File Systems(HCFS) can be used as the peristent storage layer for images, videos
    and representations. Spark is used for efficient distributed computing
    with a large set of existing libraries, that can be used in computer vision. OpenCV and Caffe
    are run on top of Spark to provide a large library of existing computer vision algorithms
    out of the box.

    In general computer vision software such as OpenCV and Caffe requires strong
    infrastructure and software deployment expertise. CloudVision removes this barrier by
    providing the platform as a service and by automating the deployment of CloudVision
    itself. CloudVision supports automated provisioning and deployment to OpenStack and AWS
    clouds. Automatic software deployment and configuration is done through Ansible.

    This paper introduces CloudVision, through in-depth review of the architecture and
    example implementations of state-of-the-art computer visions algorithms implemented
    on top of CloudVision, showing thatalgorithms implemented on CloudVision can scale horizontally
    and compare different architectures through benchmarks.

\end{eabstract}

% \ekeywords{\TeX, \LaTeX, CJK, template, thesis}
