\chapter{CloudVision架构与设计}
\label{cha:architecture}
在图\ref{fig:cloudvision-arch}可以看到CloudVision的架构主要分成两部分:
控制与管理层和业务与处理层。用户可以通过Web UI,CLI或者API使用
CloudVision的服务。 控制层提供对外的服务然后业务层负责在大数据平台
处理任务。控制与管理层和业务与处理层之间
基于一个消息队列转达任务和信息。通过分开控制层和处理层我们可以
独立扩展两部分,也保持它们独立性。如果处理性能不足可以
增加处理层的服务器。
\begin{wrapfigure}{r}{0.5\textwidth}
  \centering
  \includegraphics[width=0.48\textwidth]{cloudvision-arch}
  \caption{CloudVision 总体架构}
  \label{fig:cloudvision-arch}
\end{wrapfigure}
控制与管理层负责:
\begin{itemize}
  \item 对外提供\textbf{API}: 通过HTTP提供一个完整的RESTFUL API;
  \item 管理\textbf{云平台}: 在OpenStack或者亚马逊的AWS创建基础设施的资源;
  \item 管理\textbf{大数据处理集群}: 自动部署和搭建大数据集群在云平台上面;
  \item 管理机器视觉的\textbf{任务}: 添加用户自己写的或者执行CloudVision自带机器视觉的任务。
  \item 管理不同的\textbf{数据源}: 管理对象存储的数据源作为机器视觉任务的输入和输出。
\end{itemize}
业务与处理层主要负责执行机器视觉相关的任务:
\begin{itemize}
  \item \textbf{特征抽取}: 提供主流的特征比如CNN,SIFT和SURF。
        用户也可以自己添加新的特征抽取算法;
  \item \textbf{机器学习}: 提供基础的机器学习和分类器。
        比如K-Means,Deep Neural Networks和SVM。
  \item \textbf{抽帧}: 从多个视频抽帧然后保存到数据源或者
        分布式内存存储;
  \item \textbf{内存缓存}: 提供高效零时的存储;
\end{itemize}

在本章接下来的章节会详细介绍架构里面的每一部分。最终通过实验的方式
证明横向扩展性和不用架构的架构优势。


\section{控制与管理层}
\label{sec:arch_control}
在图\ref{fig:cloudvision-arch-control}可以看到CloudVision的
控制与管理层内部的架构。客户段通过API层提供的HTTP REST API使用
CloudVision的服务。然后CloudVision保存所有持久数据到MongoDB的数据库。
API层通过AMQP协议发信息到信息队列。然后由API workers从
信息队列收到的任务后,启动长期的业务处理的任务比如
创建虚拟机,部署集群,在集群执行机器视觉任务等等。

\begin{figure}[h]
  \centering
    \includegraphics[width=0.85\textwidth]{cloudvision-arch-control}
  \caption{CloudVision 控制与管理层架构和实现}
  \label{fig:cloudvision-arch-control}
\end{figure}

\subsection{客户端}
CloudVision提供两种不同的客户段:Web界面和Python REST Client/CLI。
Web客户段是为了让用户容易使用CloudVision的平台,不需要让用户了解细节的API接口。
Web段是基于Angular2和HTML Bootstrap框架开发,Web所有逻辑全部在浏览器运行。
由Agular2从用户的浏览器直接通过AJAX HTTP请求调用CloudVision服务器段的REST API。

\subsection{API层}
API层是跑在CloudVision的服务器段提供对外的服务。它负责
收到HTTP请求,保存数据然后把任务转到信息队列,转完了回HTTP应答。
CloudVision的RESTP API定义五种资源:Cloud(基础施舍云),Cluster(处理集群),
Job(任务),Job Template(任务模板),Datasource(数据源)。

\begin{table}[htb]
  \centering
  \begin{minipage}[t]{0.95\linewidth} % 如果想在表格中使用脚注,minipage是个不错的办法
  \caption[CloudVision API接口列表]{CloudVision API接口列表。}
  \label{tab:template-files}
    \begin{tabularx}{\linewidth}{lXlXlXlXl}
      \toprule[1.5pt]
        资源URI & GET & PUT & POST & DELETE \\
        \$base/clouds & 返回用户所有云平台。支持分页,默认提供10件。 & 没使用 & 添加用户的基础施舍云API访问信息 & None \\
        \$base/clouds/\{uuid\} & 返回所有云平台 & None &  None & None \\

      \bottomrule[1.5pt]
    \end{tabularx}
  \end{minipage}
\end{table}



\subsection{基础设施云管理}
\subsection{大数据与处理集群管理}
\subsection{机器视觉任务管理}
\subsection{数据源管理}


\section{业务与处理层}
\label{sec:arch_workers}

\section{基础设施层}
\label{sec:arch_infra}


\section{架构对比和实验}
\label{sec:arch_experiment}
