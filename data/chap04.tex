\chapter{CloudVision - 机器视觉云平台实验}
\label{cha:cloudvision_experiment}
为了证证明架构的横向扩展性和提高性能的内存缓存,本章描述
两个实验:横向扩展性实验和内存缓存性能对比实验。
所有实验基础施舍使用清华软件学院的信息系统与工程研究所OpenStack私有云,
具体用到的计算资源每个实验单独描述。持久存储使用研究所的OpenStack Swift
对象存储。


\section{横向扩展性实验}
\label{sec:scalability-experiment}
这个实验的目标是了解到CloudVision云平台不同功能的扩展性。通过增加业务与处理层的计算节点,
应该能看到一个线性的运行时间的减少。本实验使用了Caltech-256的数据集,里面
有30607图片和256个类别,总体数据大小是1.2GB。

扩展性实验分析\ref{subsec:sift-scalability} SIFT特征抽取扩展性和\ref{subsec:classifier-scalability}
分类器的分类过程扩展性。两个实验同样从1台计算节点增加到6台计算节点,分析计算时间。
每台计算服务器使用m1.large的配置,4 x vCPU,8GB内存,80GB本地硬盘。所有虚拟机共享一个
1Gbps私有网络。

\subsection{SIFT特征抽取扩展性实验}
\label{subsec:sift-scalability}
这个实验分析CloudVision实现的分布式SIFT特征抽取扩展性能力。
首先为实验把Caltech-256的数据集做成一个1.2G的SequenceFile保存
到OpenStack Swift对象存储。然后执行在\ref{sec:feature-extraction}描述CloudVision SIFT特征抽取实现,
保存SIFT特征到集群里内存缓存。

Increasing datasize and increasing workers


\subsection{分类器训练扩展性实验}
\label{subsec:classifier-scalability}



\section{内存缓存性能对比实验}
\label{sec:memory-cache-experiment}
Features to FeaturesBoW with and without Alluxio
With and without Alluxio whole pipline from SIFT to BoW with dictionary learning

How CloudVision matches NIST definition of PaaS. 
Usage of cloud storage and cloud apis
